\documentclass{article}
\usepackage{amsmath}
\usepackage{amsthm}
\usepackage{amssymb}
\usepackage{caption}
\usepackage{tikz}
\usepackage{color}
\usepackage{graphicx}
\usepackage{enumerate}
\usepackage{hyperref}
\usepackage{float}
\usetikzlibrary{matrix}

\hypersetup{
    colorlinks=true,
    linkcolor=blue,
    filecolor=blue,      
    urlcolor=blue
}

\newlength\tindent
\setlength{\tindent}{\parindent}
\setlength{\parindent}{0pt}
\renewcommand{\indent}{\hspace*{\tindent}}

\bibliographystyle{plain}

\begin{document}

\title{Estimating Reliability of EHR Data using a Dirichlet Process Prior}
\author{Jacob Carey}

\maketitle

\section{Introduction}

EHR medical data is not perfect~\cite{kanger2014}. The accuracy of individual measurements has a degree of variability, generally termed \textbf{reliability}. Reliability occurs when the "repetition of an experiment or measurement gives the same results"~\cite{porta2008}.  Typically, reliability is estimated from data with repeat measurements. However, this idea is logistically difficult to implement in EHR data, as there is usually only one set of EHR data. Instead, one approach is to use a hierarchical model and calculate the "signal to noise" of the measure~\cite{Adams2009}. \\

In our example, we simulate binomial data similar to real life presence/absence indicators for medication in EHR. We are interestd in calculating the reliability of this measure in EHR data. Here, we aggregate this binomial data by \textit{site}, a grouping of physicians in the same facility. According to~\cite{Adams2009} we calculate the $\hat{\sigma}^2_{\text{binomial}}$ for each site and the between variance. A typical model choice for this problem is \textit{Beta-Binomial model} written

\begin{equation}\label{betabinom}
\theta_i \sim \text{Beta}(\alpha, \beta)
\end{equation}

where the variance between the souces follows the definition of a Beta distribution with the approximated $\alpha, \beta$ parameters. However, the problem with \eqref{betabinom} is that we force the typically uni-modal assumption on the underylying distribution of the $\theta$'s. Such model misspecification will not influence the estimates of $\hat{\sigma}^2_{\text{binomial}}$, but it can have an effect on the approximation of the between variance. Thus, a natural extension is to use a \textit{mixture of Beta} distributions for the hierarchical distribution.

\begin{equation}\label{eq:finitemix-betabinom}
\theta_i \sim \sum_i^K pi_i \times \text{Beta}(\alpha_k, \beta_k)
\end{equation}

\eqref{eq:finitemix-betabinom} allows for a more flexible hierarchical distribution. 
\newpage

\section{Methods}

To approximate the parameters for such a Beta-Mixture Binomial model, we consider representation in Figure~\ref{fig:diagram}. In this figure, we show the relationship of the parameters. For us, the parameters of interest are those highlighted in {\color{blue}blue}. The only portion of this diagram that is actually observed is the data, highlighted in {\color{orange}orange}. The rest of the parameters are nuisance parameters invented to improve the performance of the Markov Chain Monte Carlo. \\

\begin{figure}[H]
\centering
\begin{tikzpicture}

% write symbols
\matrix[matrix of math nodes, column sep=20pt, row sep=20pt] (mat)
{
    |[blue]|\pi_k & &[2em] \mu_{\mu_k} & & \sigma^2_{\mu_k}|\mu_{\mu_k} \\
    z_i & |[blue]|\sigma^2_k | \mu_{(k)} & & |[blue]|\mu_{(k)} \\[3em]
        & |[blue]|\theta_i \\
        & |[orange]|X_i \\
};

% draw arrows
\draw[->,>=latex] (mat-1-1) -- (mat-2-1);
\draw[->,>=latex] (mat-2-1) -- (mat-3-2);
\draw[->,>=latex] (mat-3-2) -- (mat-4-2);

\draw[->,>=latex] (mat-2-2) -- (mat-3-2);

\draw[->,>=latex] (mat-1-3) -- (mat-2-4);
\draw[->,>=latex] (mat-1-5) -- (mat-2-4);

\draw[->,>=latex] (mat-2-4) -- (mat-3-2);

\end{tikzpicture}
\captionof{figure}{Diagram of Mixture-of-Beta Binomial Model}
\label{fig:diagram}
\end{figure}

Explicitly, we assume the data arrives to us as

\begin{equation}\label{eq:data}
    X_i \sim \text{Binom}(\theta_i, n_i)
\end{equation}

Next, we consider the $\theta$'s to be draws from a mixture of Beta's, each with their own $\alpha$ and $\beta$.

\begin{equation}\label{eq:theta}
    \theta_i \sim \text{Beta}(\alpha_k, \beta_k)
\end{equation}

Where the component of each $\theta_i$ is indicated by $z_i$. However, the reader will note that the diagram indicates that the Beta distributions for the $\theta_i$'s is parameterized in terms of $\mu$ (mean) and $\sigma^2$ (variance). This is a somewhat artifical construction to improve mixing. Instead we place a Beta$(1, 1)$ prior on the means and a Uniform$(0, \mu * (1-\mu))$ prior on the variance given the mean. \\

This mixture model performs very well, and reduces the uncertainty of the between variance compared to the misspecified Beta Binomial model. However, a mixture model requires an \textit{a priori} specifiction of the number of components. Sometimes, we have such prior knowledge - for example, we may know that sites have either a high probability of listing a patient on a mediciation or a low probability. But often we have no such prior knowledge. 

A solution arrives from the field of \textbf{Nonparametric Bayesian Statistics}. Instead of \eqref{eq:finitemix-betabinom}, a more relaxed choice is an \textit{infinite mixture model}

\begin{equation}\label{eq:infinitemix-betabinom}
\theta_i \sim \sum_i^{\infty} pi_i \times \text{Beta}(\alpha_k, \beta_k)
\end{equation}

In practice, an infinite mixture like \eqref{eq:infinitemix-betabinom} is implemented as a \textbf{Dirichlet Process}~\cite{gelman2004,ohlssen2007}. A Dirichlet Process can be considered as an infinite dimensional \textit{distribution over distributions} using some base probability measure. For us, we consider the mixing distribution to be a Dirichlet Process over a Beta$(1, 1)$ distribution, instead of a finite mixture of Beta distributions. While we no longer can make inference on the components as easily, we can still make inference on characteristics of the Dirichlet Process, such as the \textit{variance}, which is the only important piece for us. This approach allows us to use a flexible \textit{nonparametric} approach without forcing the data into a model that may not fit well.

\section{Results}

To assess the performance of these models, we simulated 100 observations from

\begin{equation}\label{eq:data-sim}
    \begin{split}
    \theta_i &\sim \frac{1}{2} \text{Beta}(1, 99) + \frac{3}{20} \text{Beta}(30, 70) +\frac{7}{20} \text{Beta}(45, 55) \\
    X_i &\sim \text{Binomial}(\theta_i, 100)
    \end{split}
\end{equation}

Three models were fit to this simulated data

\begin{enumerate}[1]
    \item The simple Beta Binomial
    \item The Beta Mixture Binomial model (with true $k=3$ chosen)
    \item The Dirichlet Process model
\end{enumerate}

Code for the simulation and model fit can be found at \url{https://github.com/jacobcvt12/zib}. \\

First, we compare the approximations for the between variance. In this plot, the true between variance (according to above model for simulation) is colored in orange. 95\% highest posterior density intervals for each model are colored in blue.

\begin{figure}[H]
\includegraphics{variance}
\captionof{figure}{Posterior Variance}
\end{figure}

Next, we calculate reliability according to~\cite{Adams2009}

\begin{equation}\label{eq:reliability}
    \text{Reliability}_i=\frac{\sigma^2_{\text{between}}}{\sigma^2_{\text{between}}+\hat{\sigma}^2_{\text{binomial},i}}
\end{equation}

Where $\hat{\sigma}^2_{\text{binomial},i}$ is the posterior variance of the individual binomial proportions. Note that in this definition, each aggregated site has its own estimate of reliability. We show these calculations of reliability, ordered by the point estimates. 95\% credibility estimates are colored in blue.

\begin{figure}[H]
\includegraphics{reliability}
\captionof{figure}{Reliability estimate}
\end{figure}

We see that the between variance has a much tighter highest posterior density interval comparing the Dirichlet Process model to the base model. Indeed, the between variance approximated by the relaxed Dirichlet Process model appears very similar to the Beta-Mixture model. Similarly, the credible interval of the reliability is much narrower with the Dirichlet Process than the Beta-Binomial model. \\

In this example, the semiparametric Dirichlet Process approach sees a relatively large improvement over the misspecificed Beta-Binomial Model. The reader may note that the scale of the ranges is very small, and reason that this difference is not meaningfully large. However, EHR data frequently has worse resolution than the simulated data. For example, the size of the sites may be much smaller, or the number of sites may not be as large. Due to this issue, any reduction in posterior variance through avoiding model misspecification is helpful in a real life example. Thus, the Dirichlet Process prior model provides an important alternative for calculating the reliability of EHR measures.

\bibliography{references}

\end{document}
